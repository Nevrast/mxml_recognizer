\documentclass[printmode, eng]{mgr}
\usepackage{polski}
\usepackage[utf8]{inputenc}
\usepackage[T1]{fontenc}

\usepackage{graphicx}
\usepackage{subfigure}
\usepackage{psfrag}

\usepackage{amsmath}
\usepackage{amsfonts}

\usepackage{supertabular}
\usepackage{array}
\usepackage{tabularx}
\usepackage{hhline}

\newtheorem{theorem}{Twierdzenie}[section]

\title{Projekt i implementacja aplikacji automatycznej klasyfikacji utworów muzycznych z różnych epok}
\engtitle{Design and implementation of the application for process of automatic classification of classical music from different eras}
\author{Patryk Grzybała}
\supervisor{dr inż. Maciej Walczyński, PWr}
\field{Elektronika (EKA)}
\specialisation{Inżynieria akustyczna (EIA)}

\linespread{1.3}
\begin{document}
\maketitle

\tableofcontents

\chapter{Wstęp}
\qquad W niniejszej pracy inżynierskiej omawiany jest temat automatycznego rozpoznawania utworów muzyki klasycznej z różnych epok wykorzystując do tego algorytmy uczenia maszynowego. Praca omawia wyniki uzyskane podczas tworzenia aplikacji oraz omawia napotkane problemy podczas jej tworzenia. \\

\qquad Uczenie maszynowe zyskuje na coraz większej popularności. Uczenia maszynowe jako dziedzina nauk o sztucznej inteligencji pojawiła się już w latach pięćdziesiątych ubiegłego wieku kiedy Arthur Lee Samuel po raz pierwszy, w roku 1959, użył terminu "uczenie maszynowe". Program Arthura uczył się grać w prostą grę planszową - warcaby. W kolejnych latach dzięki uczeniu maszynowemu odkryto nieznane molekuły związków organicznych, a wyniki tychże badań ukazały się w prasie naukowej i po raz pierwszy nie były to badania wykonane przez człowieka. Obecnie uczenie maszynowe jest wykorzystywane w wielu dziedzinach z ogromnymi sukcesami.
\section{Zakres pracy}
\qquad W zakres pracy wchodzą:
\begin{itemize}
\item Pozyskanie parametrów opisujących utwór muzyczny z plików MusicXML pozwalających na reprezentowanie zachodniej notacji muzycznej.
\item Stworzenie bazy danych plików muzycznych z podziałem na różne epoki
\item Wykorzystanie istniejących algorytmów uczenia maszynowego do jak najlepszej klasyfikacji utworów muzycznych
\item Stworzenie interfejsu wiersza poleceń pozwalającego na sklasyfikowanie utworu
\end{itemize}
\section{Zawartość pracy}
Rozdział 2 opisuje 
\end{document}